\documentclass[10pt]{article}
\usepackage{trabalhos-ppgmus}
\usepackage{subfloat}

\newcommand{\piece}[0]{\textit{Incessante Reflexão}}

\begin{document}
\cabecalho{Tutorial em Composição---MUS593}{Pedro Kröger}{Marcos da Silva
  Sampaio}{Relatório: Revisão da obra \eng{Incessante Reflexão}}

\section{Introdução}
\label{sec:introducao}

%% o relatório
Este relatório compreende as atividades realizadas na disciplina
Tutorial em Composição, ministrada por Pedro Kröger para mim no
semestre letivo 2010.2.

%% objetivo da disciplina
O objetivo desta disciplina foi revisar alguma uma obra minha antiga a
fim de elevar a sua qualidade composicional.
%% problema
O processo de composição de uma obra muitas vezes ocorre de forma
corrida por causa dos prazos de entrega. Dessa forma muitas vezes não
damos atenção suficiente a determinados trechos da obra, por falta de
tempo.

%% peça escolhida
Escolhi para esse trabalho a obra \piece{}, um quinteto de sopros que
compus em 2004 para o concurso de composição do IBEU
Cultural\footnote{\url{http://www.ibeu.org.br/canais/cultural/musica}},
e fiz uma revisão para submetê-la ao concurso Ernst Widmer, em
2007. Embora a peça contenha idéias interessantes, não foi
classificada em nenhum dos concursos.

\section{O processo de revisão}
\label{sec:o-processo-de}

Para realizar o trabalho Pedro e eu fizemos cada um uma revisão da
partitura e identificamos pontos frágeis da obra. Com esta revisão
concluímos que seria necessário rever articulações, esgotamento de
material, e orquestração.

A etapa seguinte e mais duradoura foi corrigir e/ou reescrever cada
trecho identificado como frágil e apresentar a Pedro para orientação.

\section{Resultados}
\label{sec:resultados}



\url{https://github.com/mdsmus/incessante-reflexao}

\section{Conclusão}
\label{sec:conclusao}

\end{document}